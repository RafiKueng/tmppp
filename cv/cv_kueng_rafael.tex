%______________________________________________________________________________________________________________________
% @brief    LaTeX2e Resume for Kamil K Wojcicki
\documentclass[a4paper,margin,line,useAMS,usenatbib]{resume}



\usepackage[T1]{fontenc}
\usepackage[sc,osf]{mathpazo}
\usepackage{hyperref}
\usepackage[english]{babel}
\usepackage{epsfig}
\usepackage{graphicx}
\usepackage{eucal}
\usepackage{latexsym}
\usepackage{fancyhdr}
\usepackage{amsfonts}
\usepackage{amssymb,amsmath}
\usepackage[utf8]{inputenc}


\pagestyle{plain}
\pagenumbering{arabic}

\linespread{1.1}


\def\gsim{~\rlap{$>$}{\lower 1.0ex\hbox{$\sim$}}}

\def\tb{\textbullet\;}

%______________________________________________________________________________________________________________________
\begin{document}
\name{\Large CURRICULUM VITAE of Rafael Küng}
\begin{resume}

%__________________________________________________________________________________________________________________
\section{\mysidestyle Personal\\Data}

First name:           \emph{Rafael}\\
Last name:            \emph{Küng}\\
Date of birth:        \emph{6 February 1984}\\
Place of birth:       \emph{Muri AG, Switzerland}\\
Home town:            \emph{Beinwil / Freiamt, Switzerland}\\
Nationality:          \emph{Swiss}

%__________________________________________________________________________________________________________________
\section{\mysidestyle Contact\\Information}

Universität Zürich                \hfill office: 36K03                              \\%\vspace{0mm}\\\vspace{0mm}%
Physik Institut                   \hfill office phone number: +41 44 635 44 06      \\%\vspace{0mm}\\\vspace{0mm}%
Winterthurerstrasse 190           \hfill e-mail: rafik$@$physik.uzh.ch              \\%\vspace{0mm}\\\vspace{-4.5mm}%
CH-8057 Z\"urich, Switzerland     \hfill webpage:
                                        \href{http://www.physik.uzh.ch/~rafik/}{http://www.physik.uzh.ch/$\sim$rafik/}



%__________________________________________________________________________________________________________________
\section{\mysidestyle Research\\interests}

  \tb Strong gravitational lensing: lenses as cosmic telescopes
  \tb Citizen science: crowd-sourced object classification and data modelling
  \tb Informatics: Data mining and analysis systems
  \tb Informatics: Networks


%__________________________________________________________________________________________________________________
\section{\mysidestyle Education\\ and career}

\begin{list2}

  \item  PhD in Physics, Universität Zürich  \hfill {\bf Feb 2014 - present}\\
  {\small Supervisor: XXX, Dr. P. Saha}

  \item MSc in Computational Science \hfill {\bf 31/12/2013}\\
  Universität Zürich \\
  {\small Thesis Title: \emph{SpaghettiLens -- Lens modeling made easy} \\
  Supervisors: Prof. G. Lake, Dr. P. Saha}.

  \item BSc in Physics \hfill {\bf 31/07/2012}\\
  Universität Zürich \\
  {\small Minors: \emph{Mathematics, Applied Informatics for Scientists, Neuroinformatics}\\
  Thesis Title: \emph{HaloViz: An analysis and visualization tool for halos in astrophysical simulations} \\
  Supervisors: Prof. B. Moore}.

  \item High school diploma \hfill {\bf Summer 2005}\\
  Kantonsschule Wohlen, Switzerland

\end{list2}


%__________________________________________________________________________________________________________________
%\section{\mysidestyle Grants}

%\begin{list2}
%  \item Forschungskredit Candoc der Universit\"at Z\"urich (No. 57181802) \hfill {\bf September 2012 - June 2013} \\ Institut f\"ur Theoretische Physik, Zurich
%  \item PhD position {\it (Swiss National Science Foundation)} \hfill {\bf Mar 2011 - September 2012} \\ Institut f\"ur Theoretische Physik, Zurich
%\end{list2}


%__________________________________________________________________________________________________________________
%\section{\mysidestyle Awards}

%\begin{list2}
%  \item Price for best Matura thesis in the Canton Aargau awarded by \hfill {\bf 2004} \\
%  \emph{Pro Argovia} and the \emph{Aargauische Naturforschende Gesellschaft}
%\end{list2}


    %__________________________________________________________________________________________________________________
    % Main formative activity
%    \section{\mysidestyle Formative\\activities}

%I attended some basic university courses about the main
%Mathematical and Physical areas: basics of Algebra and Geometry
%and Mathematical Analysis, Kinematics, Dynamics, Thermodynamics,
%Fluidodynamics, Quantum Physics, Particle and Nuclear Physics,
%Solid State Physics, Special and General Relativity, theoretical
%and observational Astrophysics. During the PhD, I concentrated on
%Cosmology (``dark energy'' and ``dark matter'') and Astrophysics.
%
%Doctoral courses:
%\begin{list2}
%\item Theory of signals and systems (L. Milano);
%\item Observational cosmology (G. Longo, C. Rubano and G. Miele);
%\item Theoretical cosmology (G. Miele);
%\end{list2}

%Schools and conferences:
%\begin{list2}
%\item ``Galaxy Evolution and Environment 2 (GEE2)'', 7-9 November 2011, Milano, Italy;
%\item ``ESO - Fornax, Virgo, Coma et al.'', 27 June - 1 July 2011,
%Garching, Munich, Germany;
%\item ``III Italian-Pakistani workshop on relativistic astrophysics'', 20-22
%June 2011, Lecce, Italy;
%\item ``Salerno Microlensing Conference'', 18-22 January 2011,
%Salerno, Italy;
%\item ``Metal enrichment from Hydrodynamical simulation workshop'', 16-17
%September 2010, El Escorial, Madrid, Spain;
%\item ``L'astronomia italiana: prospettive per la prossima
%decade'', 54$^{o}$ Congresso SAIt, 4-7 May 2010, INAF-Osservatorio
%Astronomico di Capodimonte, Napoli, Italy;
%\item ``Galaxies Properties Across Cosmic Ages'', 28-29 April
%2009, Accademia dei Lincei, Rome, Italy;
%\item ``Oxford-COSMOCT Workshop on the Interface between Galaxy Formation and AGNs'',
%18-22 May 2008, Vulcano, Messina, Italy;
%\item ``Scuola Nazionale di Astrofisica (VIII ciclo - III
%corso)'', \emph{Dinamica delle galassie - Nuclei galattici
%attivi}, 7-12 May 2006, Bertinoro, Forl\'i-Cesena, Italy;
%\item ``$I^{0}$ Workshop di Astronomia ed Astrofisica per studenti'', 19-20 April
%2006, Napoli, Italy;
%\item ``Scuola Nazionale di Astrofisica (VIII ciclo - I corso)'', \emph{Cosmologia
%osservativa a grande campo - Scala delle Distanze}, 8-13 May 2005,
%S. Agata sui Due Golfi, Napoli, Italy;
%\item $39^{th}$ Rencontres de Moriond: ``Exploring the Universe'',
%28 March - 4 April 2004, La Thuile, Valle d'Aosta, Italy;
%\item ``Thinking, Observing and
%Mining the Universe'', 22/27 September 2003, Sorrento, Napoli,
%Italy;
%\item ``$33^{rd}$ Saas-Fee Course in Gravitational Lensing'', 5/12 April
%2003, Les Diablerets, France
%\end{list2}





%__________________________________________________________________________________________________________________
\section{\mysidestyle Language and\\ computing skills}

{\it Languages:}
\begin{list2}
  \item German       \hfill {\bf mother tongue}
  \item English      \hfill {\bf fluent in speaking and writing}
  \item French       \hfill {\bf good}
\end{list2}

{\it Software skills:}\\
Extensive use of the \tb GNU/Linux and \tb Windows operating systems, both as administrator and user. Good knowledge of \tb \LaTeX, \tb Office and \tb Matlab.

{\it Programming:}\\
Good knowledge of \tb C++ \tb Python \tb Java \tb HTML \tb JavaScript and \tb PHP.

%    %__________________________________________________________________________________________________________________
%    % contracts
%    \section{\mysidestyle Job contracts}
%
%\begin{list2}
%\item Post-doctoral position {\it (Swiss National Science Foundation)} \hfill {\bf Oct 2009 - present} \\ Institut f\"ur Theoretische
%Physik, Zurich
%\item Grant {\it (Project Mecenass)} \hfill {\bf Feb 2009 - Aug 2009} \\ Universit\`{a} di Napoli Federico II
%\item Co.Co.Pro. contract {\it (Consorzio Cometa)} \hfill {\bf Feb 2008 - Oct 2008} \\
%Dipartimento di Fisica ed Astronomia - Universit\`a di Catania.
%\item Research contract \hfill {\bf May 2006 - Apr 2007}\\
%Osservatorio Atronomico di Capodimonte, Napoli.
%\item Grant \hfill {\bf Oct 2005 - May 2006}\\
%Universit\`{a} di Napoli Federico II
%\item Co.Co.Co. contract \hfill {\bf Jul 2004 - Jun 2005}\\
%Universit\`a di Salerno, Italy.
%\item Contract of occasional collaboration \hfill {\bf Apr 2004 -
%Jul 2004}\\
%INFM (Napoli) in order to realize a multimedial hypertext \\
%(experiments about interference and coherence of light), \\ with
%the supervision of Prof. E. Santamato.
%\end{list2}



%__________________________________________________________________________________________________________________
\section{\mysidestyle Didactical\\activity}

\begin{list2}

  \item Tutor and lecture assistant for the lectures \emph{Physik I/II}
        \hfill {\bf 2013/2014} \\
        held by Prof. U.~Straumann
        Universität Zürich

  \item Tutor for the colloquium \emph{Physik für Mediziner}
        \hfill {\bf 2009 -- 2012} \\
        held by Prof. J. Osterwalder\\
        Universität Zürich

  \item Tutor for the colloquium \emph{Physik für Biologen}
        \hfill {\bf 2009 -- 2012} \\
        held by Dr. Ch. Aegerter\\
        Universität Zürich

  \item Highschool teacher for \emph{Physics}
        \hfill {\bf Jul 2011 - Dec 2011}\\
        Realgymnasium Rämibühl, Zürich

  \item Highschool teacher for \emph{Physics}
        \hfill {\bf Jan 2009 - Jul 2011}\\
        Literargymnasium Rämibühl, Zürich

\end{list2}



\newpage

%__________________________________________________________________________________________________________________
%\section{\mysidestyle Conferences and Schools}
%
%\begin{list2}
  %\item \emph{LISA Workshop on Massive Black Hole Binaries in the  \hfill{\bf Feb 10-12 2010} \\
      %Cosmic Landscape}\\
      %Universit\"at Z\"urich, Switzerland
  %\item \emph{XLVI$^{\mbox{\scriptsize th}}$ Rencontres de Moriond - Gravitational Waves and \hfill{\bf Mar 13-27 2011} \\
    %Experimental Gravity} \\
    %La Thuile, Italy
  %\item \emph{School On Gravitational Waves - from Theory to Detection} \hfill {\bf May 23-28 2011}\\
    %Institut d'Etudes Scientifiques de Carg\`{e}se, France      
  %\item \emph{AstroGR@Mallorca} \hfill {\bf Sep 5-9 2011} \\
    %Universitat de les Illes Balears, Palma de Mallorca, Spain
  %\item \emph{9th LISA Symposium} \hfill {\bf May 21-25 2012} \\
    %Biblioth\`{e}que Nationale de France, Paris, France
  %\item \emph{AstroGR@Beijing} "Black Hole Growth in the Universe" \hfill {\bf Sep 3-7 2012}\\
    %National Astronomical Observatories of Chinese Academy of Sciences, Beijing, China
  %\item \emph{First eLISA Consortium Meeting} \hfill {\bf October 22-23 2012} \\
    %APC Laboratory, Paris, France
%\end{list2}




%__________________________________________________________________________________________________________________
\section{\mysidestyle Main\\collaborators}

\begin{list2}
  \item Dr. P. Saha (Department of Physics, UZH)
  \item Dr. P. Marshall (Kavli Institute for Particle Astrophysics and Cosmology, Stanford University)
  \item Dr. A. More (Kavli IPMU, University of Tokyo)
  \item Dr. A. Verma (Department of Physics, University of Oxford)
  \item Dr. J. R. Brownstein (Department of physics and astronomy, University of Utah)
\end{list2}



%__________________________________________________________________________________________________________________
\section{\mysidestyle Publications}

Published papers:

\begin{list2}
  \item
    {\small \emph{\href{http://arxiv.org/abs/1234.1234}{``Lens Modeling in SpaceWarps''}}}\\
    {\bf Rafael Küng}, Other Authors
    2014,
    Physical Review D, volume 86, issue 8, 084028
  %\item {\small \emph{\href{http://arxiv.org/abs/1006.1664}{``Gravitational Waves from Intermediate-mass Black Holes in Young Clusters''}}}\\ M. Mapelli, {\bf C. Huwyler}, L. Mayer, Ph. Jetzer \& A. Vecchio 2010, The Astrophysical Journal, volume 719, 987
  %\item {\small \emph{\href{http://arxiv.org/abs/1108.1826}{``Testing General Relativity with LISA including Spin Precession and Higher Harmonics in the Waveform''}}}\\ {\bf C. Huwyler}, A. Klein \& Ph. Jetzer 2011, Physical Review D, volume 86, issue 8, 084028
\end{list2}






%______________________________________________________________________________________________________________________
\end{resume}
\end{document}


%______________________________________________________________________________________________________________________
% EOF
