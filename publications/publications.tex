\documentclass[12pt]{article}

\usepackage{hyperref}
\usepackage[utf8]{inputenc}


\def\ap#1#2#3{     {\it Ann. Phys. (NY) }{\bf #1} (19#2) #3}
\def\arnps#1#2#3{  {\it Ann. Rev. Nucl. Part. Sci. }{\bf #1} (19#2) #3}
\def\ijmp#1#2#3{   {\it Int. J. Mod. Phys. }{\bf #1} (19#2) #3}
\def\jhep#1#2#3{   {\it J.High Energy Phys. }{\bf #1} (19#2) #3}
\def\npb#1#2#3{    {\it Nucl. Phys. }{\bf B\,#1} (19#2) #3}
\def\plb#1#2#3{    {\it Phys. Lett. }{\bf B\,#1} (19#2) #3}
\def\prd#1#2#3{    {\it Phys. Rev. }{\bf D\,#1} (19#2) #3}
\def\prep#1#2#3{   {\it Phys. Rep. }{\bf #1} (19#2) #3}
\def\prl#1#2#3{    {\it Phys. Rev. Lett. }{\bf #1} (19#2) #3}
\def\ptp#1#2#3{    {\it Prog. Theor. Phys. }{\bf #1} (19#2) #3}
\def\rmp#1#2#3{    {\it Rev. Mod. Phys. }{\bf #1} (19#2) #3}
\def\zpc#1#2#3{    {\it Zeit. f\"{u}r Physik }{\bf C\,#1} (19#2) #3}
\def\epjc#1#2#3{   {\it Eur. Phys. J. }{\bf C\,#1} (19#2) #3}
\def\mpla#1#2#3{   {\it Mod. Phys. Lett. }{\bf A\,#1} (19#2) #3}
\def\sjnp#1#2#3{   {\it Sov. J. Nucl. Phys. }{\bf #1} (19#2) #3}
\def\yf#1#2#3{     {\it Yad. Fiz. }{\bf #1} (19#2) #3}
\def\nc#1#2#3{     {\it Nuovo Cim. }{\bf #1} (19#2) #3}
\def\jetpl#1#2#3{  {\it JETP Lett. }{\bf #1} (19#2) #3}
\def\ib#1#2#3{     {\it ibid. }{\bf #1} (19#2) #3}
%----------------------------------------------------------------------



\setlength{\parskip}{2ex}
\setlength{\parindent}{2em}
\setlength{\textwidth}{16cm}
\setlength{\textheight}{22.5cm}
\setlength{\oddsidemargin}{0.25cm}
\setlength{\evensidemargin}{0.25cm}
%\setlength{\topmargin}{-2.0cm}         % Zurich
\setlength{\topmargin}{-1.0cm}         % Bern
%\def\thefootnote{\fnsymbol{footnote}}

\setlength{\parindent}{0pt} % No indent

\renewcommand{\thesection}{\Alph{section}}
\renewcommand{\thesubsection}{\arabic{subsection}}

%-----------------------------------------------------------------------

\begin{document}
\pagestyle{empty}

\section*{List of Publications}

\subsection{Published Papers}
Currently, I don't have any published papers.

\subsection{In Preparation}

\begin{itemize}
\item
  \emph{``Lens Modeling in SpaceWarps''}\\
  {\bf Rafael Küng}, Philip Marshall, Anupreeta More, Aprajita Verma, Amit Kapadina, Elisabeth Baeten, Claude Cornen, Christine McMillian, Julianne Wilcox, Jonas Odermatt, Jonathan Coles, Surhud More and Prasenjit Saha\\
  {\small
  to be available in Feb, 2014\\
  Evaluation of volunteers modelling performance using SpaghettiLens. The follow up paper to my MSc thesis.
  (Live version available (beta)\footnote{\url{http://mite.physik.uzh.ch}}.
  Documentation (MSc thesis) available\footnote{\url{http://www.physik.uzh.ch/~rafik/docs/spaghettilens.pdf}}.
  Source code available\footnote{\url{https://github.com/RafiKueng/SpaghettiLens}}.)
  }
  
\item
  \emph{``An All-Inorganic Responsive Surface: Reversible Electrochemical Contact Angle Switching of Hexagonal Boron Nitride Nanomesh''}\\
  Stijn F.L. Mertens, Stefan Muff, Adrian Hemmi, {\bf Rafael Küng}, Steven De Feyter, Jürg Osterwalder and Thomas Greber\\
  {\small
  Abstract submitted April 2012 for ECOSS2012, currently pending\\
  I implemented an algorithm for automatic contact angle measurement of a fluid on a monolayer of hexagonal boron nitride on Rh(111).
  (Source code available\footnote{\url{https://github.com/RafiKueng/dropContactAngleMeasurement}}.)
  }
\end{itemize}

\subsection{Theses}

\begin{itemize}
  \item
    \emph{``SpaghettiLens -- Lens modeling made easy''}
    \footnote{\url{http://www.physik.uzh.ch/~rafik/docs/spaghettilens.pdf}}\\
    MSc thesis in Computational Science at UZH, September 2013
  \item
    \emph{``HaloViz --  An analysis and visualization tool for halos in astrophysical \\ ssimulations''}
    \footnote{\url{http://www.physik.uzh.ch/~rafik/docs/haloviz.pdf}}\\
     BSc thesis in Physics at UZH, March 2011
\end{itemize}


\end{document}
